\Chapter{Tesztelés}

\Section{Reguláris kifejezések}

Regluráis kifejezések illesztési algoritmus gyakran eltérő, a szerzők másként értelmeznek bizonyos eseteket, vagyis nincs egységes szabvány belőle.
Ezért ennek ellenőrzésére konzulensemmel készítettünk egységtesztet Cmockában. Amennyiben a tesztelésünk sikeres volt és a reguláris kifejezés sablonunk megfelelően illeszkedik az adott sorra abban az esetben teszt futtatás után a következő kimenetet kell kapjuk:
\begin{python}
[==========] Running 6 test(s).
[ RUN      ] test_vertex_line
[       OK ] test_vertex_line
[ RUN      ] test_texture_vertex_line
[       OK ] test_texture_vertex_line
[ RUN      ] test_vertex_normal_line
[       OK ] test_vertex_normal_line
[ RUN      ] test_face_line
[       OK ] test_face_line
[ RUN      ] test_triangle_line
[       OK ] test_triangle_line
[ RUN      ] test_quad_line
[       OK ] test_quad_line
[==========] 6 test(s) run.
[  PASSED  ] 6 test(s).
\end{python}
\noindent Abban az esetben, ha a tesztünk valamelyik fázisban elbukott, akkor a következővel tér vissza a teszt:
\begin{python}
[ RUN      ] test_quad_line
[  ERROR   ] --- result
[   LINE   ] --- tests/regex_tests.h:72: error: Failure!
[  FAILED  ] test_quad_line
[==========] 6 test(s) run.
[  PASSED  ] 5 test(s).
[  FAILED  ] 1 test(s), listed below:
[  FAILED  ] test_quad_line
\end{python}
Ennél a példánál a \texttt{QUAD\_LINE\_PATTERN} nem megfelelően illeszkedett  a tesztben megadott formátumra, ezért hibával tér vissza.
\Section{Modell betöltés sebességmérés}

\begin{figure}[h]
\centering
\includegraphics[width=\textwidth]{images/betoltesiido.png}
\caption{Modell betöltési idő.}
\label{fig:betolt}
\end{figure}

Modellek betöltési sebességét vizsgáltam a  \ref{fig:betolt}. grafikonon. Az X tengelyen a modellek összes elemszáma (geometria vektor, textúra koordináta, normál vektor,arc elem) számok összesítése látható az Y tengelyen pedig a betöltési idő milliszekundumban. Az összehasonlítás textúra, modell kirajzolás, és az OBJ fájlon történő bármilyen változtatás nélkül történt.\\

A grafikonon jól látható hogy az idő növekedése közel exponenciális, tehát a modellek elemszámának növekedésének mértéke közel  arányos a beolvasáshoz szükséges idő nagyságával.


\Section{Modell betöltés sebességmérés az egyes részeken}