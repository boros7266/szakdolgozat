\Chapter{Javító szoftver megvalósítás}
A szoftver egy C nyelvű csomagként került megvalósításra, \texttt{obj\_corrector\_tool} (Objektum Javító Eszköz) névvel ellátva. A mappa tartalmazza az összes fájlt a programhoz illetve az egységtesztekhez.

Csomag struktúrális felépítése a következő:
\bigskip
\dirtree{%
.1 /obj\_corrector\_tool.
.2 /include.
.3 bounding\_box.h.
.3 draw.h.
.3 model.h.
.3 \ldots.
.2 /OBJ.
.3 bird.
.4 bird.jpg.
.4 bird.obj.
.4 bird.mtl.
.4 \ldots.
.3 cube.
.4 cube.png.
.4 cube.obj.
.4 cube.mtl.
.3 \ldots.
.2 src.
.3 bounding\_box.c.
.3 draw.c.
.3 model.c.
.3 main.c.
.3 \ldots.
.2 tests.
.3 regex\_tests.h.
.3 test\_runner.c.
.2 Makefile.
}
\newpage
Az \texttt{Include} mappa tartalmazza a szükséges fejléc állományokat. Az \texttt{OBJ} almappáiban találhatóak a felhasznált OBJ fájlok. Az \texttt{src}-ben találhatóak a C nyelvű forrásfájlok. \texttt{Tests} mappában találhatók a CMocka könyvtárral készített egységtesztek. Végül, közvetlen az \texttt{obj\_corrector\_tool} található a \texttt{Makefile}.

\Section {Funkció hívás diagram}

\Aref{fig:funk} diagramon látható a program teljes funkcióhívás diagramja.

\begin{figure}[h]
	\centering
	\includegraphics[width=\textwidth]{images/func.png}
	\caption{Teljes Funkcióhívás Diagram.}
	\label{fig:funk}
\end{figure}

Ez a diagram nagyon komplex és rengetek különálló funkciót kezel egyben. A különböző függvények egymást hívják meg a program futattása során, végül az összes főbb funkció a \texttt{main} függvényben találkozik.
Értelemszerűen, mivel egy OBJ javító alkalmazásról lévén szó, az egyes funkciókat érdemes együtt  kezelni. A program főbb funkciói:
 \begin{itemize}
\item modell beolvasás,
\item textúra beolvasás,
\item hibák javítása,
\item fájlba írás,
\item megjelenítés.
\end{itemize}
A feldolgozási folyamatot fentről-lefelé értelmezhetjük.
A következő szakaszok ezeket tekintik át.

\Section{Modell beolvasás}

Az alkalmazás minden olyan OBJ fájl betöltésére alkalmas, ami nem tartalmaz négyszögnél nagyobb csúcsszámmal rendelkező sokszöget. Amennyiben az OBJ fájlunk tartalmaz ennél magasabb számút, azok számát képes összeszámlálni, viszont kezelni nem.

Elemzésre és javításra szánt OBJ fájl betöltése \texttt{main.c} (C nyelven írt programfájlon) belül a \texttt{load\_model} függvénnyel történhet:
\begin{cpp}
load_model("utvonal/pelda.kiterjesztes", &model, &regular);
\end{cpp}
Az \texttt{utvonal}-nál meg kell adni a fájl elérési útvonalát, majd a \texttt{pelda} helyére az OBJ fájlunk nevét majd a kiterjesztést (\texttt{.obj}) írjuk.

\SubSection{A \texttt{Model} struktúra feltöltésének lépései}

A \texttt{load\_model} függvény beolvassa a megadott OBJ fájlt.
\begin{figure}[h]
\centering
\includegraphics[scale=0.5]{images/load.png}
\end{figure}
\bigskip

Beolvasás után \texttt{count\_elements} összeszámolja \texttt{model} struktúra \texttt{n\_vertices},  \texttt{\\n\_texture\_vertices}, \texttt{n\_normals}, \texttt{n\_faces} illetve \texttt{n\_triangles}, \texttt{n\_quads} és \\ \texttt{n\_\-polygons} egyes értékeit.
\begin{figure}[h]
\centering
\includegraphics[scale=0.5]{images/count.png}
\end{figure}

Ezt követően az összeszámolt értékek számával \texttt{create\_arrays} lefoglalja a szükséges nagyságú tömböket.
\begin{figure}[h]
\centering
\includegraphics[scale=0.5]{images/create.png}
\end{figure}

Ez után a lefoglalt tömböt a \texttt{read\_elements} kiolvassa a fájlból, és feltölti az egyes értékekkel.
\begin{figure}[h!]
\centering
\includegraphics[scale=0.5]{images/read.png}
\end{figure}

\newpage

\Section{Textúrák kezelése}

A textúra pontok az $(u, v) \in [0, 1]^2$ formában adhatók meg. Az intervallum az OBJ fájlok egy részében nem így van. Ennek ellenőrzésére a  \texttt{print\_texture\_box} függvény nyújthat segítséget. Amennyiben utasítjuk a programot  a \texttt{texture\_box} kiszámítására és megjelenítésére, abban az esetben egy fix struktúrát jelenít meg számunkra a program, és benne jelzi a számított adatokat, helyes esetben a következő formában
\begin{python}
Texture box:
u in [0.000000, 1.000000]
v in [0.000000, 1.000000]
\end{python}
\bigskip
Balról jobbra haladva először megjeleníti a fájlban \textbf{u} legkisebb és legnagyobb értékét, majd ugyanezt megvalósítja a \textbf{v} koordináta számított adataival is.
Az elkészült szoftver 2D (kép formátumú) textúrák megjelenítését támogatja. A kívánt fájl megadása a \texttt{texture.c} forrásfájlon belül a \texttt{texture\_filename} után lehetséges.
\begin{cpp}
char texture_filename[] = "utvonal/pelda.kiterjesztes";
\end{cpp}
Saját képfájlunk megadásánál \texttt{utvonal} helyére a fájlunk elérési útvonala \texttt{pelda} helyére a képfájlunk neve \texttt{kiterjesztes} helyére pedig kiterjesztés kerül (\texttt{.jpg , .png, .tga}).

\Section{Modell megjelenítése}

A modellek mérete és elhelyezkedése nagy mértékben eltérő lehet. Ennek elemzése érdekében érdemes megnézni a \texttt{print\_bounding\_box} által számított adatokat és a kamerát ennek megfelelően pozíciónálni.
\bigskip
\begin{python}
Bounding box:
x in [-0.500000, 0.500000]
y in [-0.500000, 0.500000]
z in [-0.500000, 0.500000]
\end{python}
\bigskip
Ezen számok megadják, hogy az objektumunk mérete mekkora lesz a térben. Minden koordináta első száma a legkisebb második száma pedig a legnagyobb csúcs értékének elhelyezkedését adja az adott tengely szerint.
A kamerapozíció beállítása ezt követően az alábbi kódrész formájában történhet
\begin{cpp}
gluLookAt
(
    0.0, 0.0, -200, // eye (X, Y, Z)
    0.0, 0.0, 0.0,  // center (X, Y, Z)
    0.0, 1.0, 0.0   // up (X, Y, Z)
);
\end{cpp}
A Kamera pozíciót a \texttt{gluLookAt} paraméterei állításával lehet pozícionálni.
<<<<<<< HEAD

Javító szoftverről lévén szó, maga a megjelenítő minimális funkciókkal lett ellátva, tehát nem tudjuk mozgatni  a modellünket, illetve kamera pozíciója sem változtatható megjelenítés után. Egyedül a modell forgatása lehetséges egér segítségével.

=======

Javító szoftverről lévén szó, maga a megjelenítő minimális funkciókkal lett ellátva, tehát nem tudjuk mozgatni  a modellünket, illetve kamera pozíciója sem változtatható megjelenítés után. Egyedül a modell forgatása lehetséges egér segítségével.

>>>>>>> d90ab7ad850e6073308db2fb32cf995d4f8fdada
A \texttt{house.obj} megjelenítése a hozzátartozó textúrával \aref{fig:demo}. ábrán látható.
\bigskip

\begin{figure}[h]
\centering
\includegraphics[width=\textwidth]{images/demo.png}
\caption{House.obj modell megjelenítés.}
\label{fig:demo}
\end{figure}
<<<<<<< HEAD
\newpage
=======

>>>>>>> d90ab7ad850e6073308db2fb32cf995d4f8fdada
\Section{Javítható hibák}

\SubSection{Háromszögesítés}

Korábbi észrevételeim alapján egyes betöltőknél gondot okoz, a háromszögeknél több csúcsszámú sokszögek kezelése. Ennek a problémának a javítására fejlesztettem ki a \texttt{triangulation\_of\_quads} funkciót. A funkció célja, hogy a négyszögeket két különálló  háromszögre bontsa szét.

\begin{figure}[h]
\centering
\includegraphics[scale=0.39]{images/triangulation.png}
\caption{Háromszögesítés elméletben.}
\label{fig:tri}
\end{figure}

A célünk ezzel, a \ref{fig:tri}. ábrán  látható módon minden négyszöget felbontsunk két különböző háromszögre a szaggatott vonalon való metszéssel. Persze, az ábrán látható négyszög egy szabályos négyzet, de ezek a négyszögek lehetnek általánosak.
A felbontás úgy valósul meg, hogy a program kiválasztja az első beolvasott négyszöget az OBJ fájlból annak első három csúcs koordinátájából képez egy háromszöget, majd a következő háromszöget az előző négyszög első harmadik, illetve negyedik koordinátájából képezi le így végighaladva az összes négyszögön ami az OBJ fájlunkban található.\\Ennek szemléletetése a \ref{fig:tri1}. ábrán látható.

\begin{figure}[h]
\centering
\includegraphics[scale=0.39]{images/haromszog.png}
\caption{Háromszögesítés diagram.}
\label{fig:tri1}
\end{figure}

A megvalósítás során \texttt{triangulation\_of\_quads} funkció modell struktúránkból elveszi az összes négyszöget és két háromszöggel tölti fel azok helyét, illetve a lapok számát is növeli minden felvágott négyszög esetében eggyel.

\begin{figure}[h]
\centering
\includegraphics[scale=0.32]{images/helmet1.png}
\includegraphics[scale=0.32]{images/helmet2.png}
\caption{helmet.obj háló.}
\label{fig:tri2}
\end{figure}

Az \ref{fig:tri2}. ábra bal oldali képen az eredeti \texttt{helmet.obj} modell látható, a modell hálójának megjelenítésével, a jobb oldali képen pedig már a háromszögelt változat. Jól láthatóak az eltérések a két modell rajzolt állapota között.

\begin{table}[h]
\centering
\caption{helmet.obj adatait tartalmazó táblázat}
\bigskip
\label{tab:modellek}
\begin{tabular}{l|c|c|}
& helmet.obj & helmet.obj háromszögesítés után \\
\hline
geometria csúcsok & 6066 & 6066 \\
textúra koordináták & 6409 & 6409 \\
normál csúcsok & 6650 & 6650 \\
lap elemek & 6064 & 12128 \\
háromszögek & 0 & 12128 \\
négyszögek & 6064 & 0 \\
\hline
\end{tabular}
\label{fig:tri3}
\end{table}

\Aref{fig:tri3}. összesítő táblázat megmutatja hogyan változtak a \texttt{helmet.obj} adatai a háromszögesítés után. Láthatóan az összes négyszög eltünt a modellből ezeket két háromszögre bontotta a szoftver, az lap elemek száma is duplájára nőtt.
<<<<<<< HEAD
\newpage
=======

>>>>>>> d90ab7ad850e6073308db2fb32cf995d4f8fdada
\SubSection{Bejárási irány megváltoztatás}

A program ezen része arra szolgál, hogy a háromszögeink bejárási irányát meg tudjuk változtatni \aref{fig:bej1}. ábrán látható módon.

\begin{figure}[h]
\centering
\includegraphics[scale=0.5]{images/bejarasi.png}
\caption{Bejárási irány változtatás elméletben.}
\label{fig:bej1}
\end{figure}

Egy háromszög csúcsok közötti bejárást kétféleképpen tudjuk megad. \Aref{fig:bej1}. ábrán bal oldalt látható az óramutató járásával ellentétes bejárási sorrendben ($1 \rightarrow 3 \rightarrow 2$), amit a megjelenítőnk alapvetően használ, jobb oldalon pedig az ellenkezője, az óramutató járásával megegyező bejárás ($1 \rightarrow 2 \rightarrow 3$).

Ez a funkció \texttt{change\_vertex\_order} függvény néven van a programban implementálva. Amennyiben úgy döntünk, és ezt alkalmazzuk a programunk elvégzi a bejárási irány cserét model struktúrába betöltött objektomunkon.
<<<<<<< HEAD

Megjelenítés során a megjelenítő ugyanúgy óramutatójárásval ellentétesen fogja megjeleníteni a modellünket, viszont a programunk \texttt{change\_vertex\_order} függvény segítségével megváltoztatta csúcsok sorrendjét így a megjelenítés már óramutató járásával megegyező módon fog történni az eredeti OBJ fájlhoz képest.
\newpage
=======

Megjelenítés során a megjelenítő ugyanúgy óramutatójárásval ellentétesen fogja megjeleníteni a modellünket, viszont a programunk \texttt{change\_vertex\_order} függvény segítségével megváltoztatta csúcsok sorrendjét így a megjelenítés már óramutató járásával megegyező módon fog történni az eredeti OBJ fájlhoz képest.

>>>>>>> d90ab7ad850e6073308db2fb32cf995d4f8fdada
\Aref{fig:bej2}. ábra egy egyszerű modellen keresztül ezt hivatott bemutatni.

\begin{figure}[h]
\centering
\includegraphics[scale=0.5]{images/order.png}
\caption{Bejárási irány hiba.}
\label{fig:bej2}
\end{figure}

Jól látható módon a felső modellünk hibásan, az alsó pedig jól került megjelenítésre. A két modell összesen az lapok közötti bejárási sorrend a különbség, az összes többi csúcsa ugyanaz. \Aref{fig:bej3} táblázatban bemutatásra kerül a két objektum közötti eltérés. (A táblázat csak a lapok közötti eltérést mutatja.)

\begin{table}[h]
\centering
\caption{Lap elemeket összehasonlító táblázat}
\bigskip
\begin{tabular}{|c|c|}
Felső modell& Alsó modell \\
\hline
f  1//1 2//2 3//3 & f  1//1 3//3 2//2 \\
f  1//1 3//3 4//4 & f  1//1 4//4 3//3 \\
f  5//5 6//6 7//7 & f  5//5 7//7 6//6 \\
f  5//5 7//7 8//8 & f  5//5 8//8 7//7 \\ 
f  1//1 2//2 6//6 & f  1//1 6//6 2//2 \\
f  1//1 5//5 6//6 & f  1//1 6//6 5//5 \\
f  1//1 4//4 8//8 & f  1//1 8//8 4//4 \\
f  1//1 5//5 8//8 & f  1//1 8//8 5//5 \\
f  3//3 6//6 7//7 & f  3//3 7//7 6//6 \\
f  2//2 3//3 6//6 & f  2//2 6//6 3//3 \\
f  3//3 8//8 7//7 & f  3//3 7//7 8//8 \\
f  3//3 4//4 8//8 & f  3//3 8//8 4//4 \\
\hline
\end{tabular}
\label{fig:bej3}
\end{table}
\bigskip 
\newpage
\Aref{fig:bej3} táblázatból láthatóan kivehető, hogy a háromszögünk lap felsorolásának iránya megváltozott a második és harmadik koordinátán, ezáltal a megjelenítés óramutató járásával megegyezően fog történni az eredeti OBJ fájl-hoz képest.

\Section{Fájlba írás}

A javított OBJ fájl kiírása az \texttt{obj\_output.obj} fájlba a \texttt{write\_to\_file} függvény segítségével kivitelezhető. OBJ fájlunk betöltése, és az elvégzett változtatások után a program megkérdezi, hogy kívánjuk menteni a változtatott OBJ fájlt, amennyiben így döntünk a program létrehozza az \texttt{obj\_output.obj} fájlt és azt feltölti model struktúránk aktuális adataival. A mentett fájlunkba kiírt OBJ fájl nem tartalmazza a betöltött fájl anyagjellemzőit illetve a megjegyzéseit sem.
\bigskip
\dirtree{%
.1 /obj\_corrector\_tool.
.2 /include.
.2 /OBJ.
.2 /src.
.2 /tests.
.2 /Makefile.
.2 /obj\_output.obj.
}
\bigskip

Ez a kimeneti fájl közvetlen az \texttt{obj\_corrector\_tool} könyvtárba kerül eltárolásra a program futtatása után.
