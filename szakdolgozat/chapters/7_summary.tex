\Chapter{Összefoglalás}
A dolgozatban egy saját fejlesztésű objektum javító szoftver került elkészítésre és bemutatásra. Ezt megelőzően áttekintésre kerültek grafikai megjelenítéssel kapcsolatos alapfogalmak.

Az alkalmazás egy \texttt{obj\_corrector\_tool} nevű C nyelvű függvénykönyvtárban került megvalósításra.

Szakdolgozat témaválasztás során nem véletlenül választottam ezt a témát, egyetemi tanulmányaim alatt már rájöttem, hogy a grafikus dolgok érdekelnek a leginkább az informatikában ezen belül is a 3D-s grafika.

Az adott feladatot, miszerint egy saját javító szoftvert hozzak létre nagyon élveztem, és sokat tanultam más szoftverek tesztelésétől elkezdve mind a tervezésen keresztül a megvalósításig végzett lépéseknél.
A program végül maga egyszerűségével maximálisan elérte a tervezett célokat. Számomra egy mérföldkő, mivel ilyen komplex önálló feladatot az egyetemen még nem kellett létrehoznom.

Továbbfejlesztési lehetőségek közül talán a legfontosabb, és egyben leginkább kézen fekvő, hogy ne csak háromszögeket és négyszögeket kezeljen a szoftver hanem ennél nagyobb csúcspont számú sokszögeket is, emelett pedig több további javítási funkció implementálása is célszerű lehet. Nem utolsó sorban a teljes program tesztelés elkészítése a hibátlan működés ellenőrzéséhez.
