\Chapter{Koncepció}
% TODO: Le kellene írni az OBJ formátumról a fontosabb dolgokat!
\Section{OBJ formátum}

Az OBJ fájlformátumot elsőnek a Wavefront Technologies kezdte kifejleszteni az 1980-as években az Advanced Visualizer animációs csomagjához. Ez a formátum egy olyan karakteres fájlformátum, ami utat enged geometriai objektumok leírásának egy aránylag egyszerű és egységes formában. \\
Magát a formátumot ott logikus használni, ahol egyszerű modelleket jelenítünk meg és ezeket nem akarjuk animálni. Például egy olyan egyszerű autó szimulátoros játéknál, ahol maga az autó geometriáját tekintve nem változik maximum a pozíciója és az orientációja, ezeket a transzformációs mátrix állításával egyszerűen megadhatjuk, így nincs szükség a modell animálására.\\
Wavefront formátumbemutatása egy egyszerű példával, ez a példa egy szimpla négyszöget definiál:\\

\begin{cpp} 
v 0.0 0.1 0.0
v 1.0 0.0 0.0
v 1.0 1.0 0.0
v 0.0 1.0 0.0

vt 0.0 0.0
vt 1.0 0.0
vt 1.0 1.0
vt 0.0 1.0

vn 1.0 0.0 0.0 

f 1/1/1 2/2/1 3/3/1 4/4/1
\end{cpp}

A fájl először felsorolja a csúcspontokat \textbf{v} (vertex), utána a textúra pontokat \textbf{vt} (vertex texture) és a normálvektorokat \textbf{vn} (vertex normal) utána  az \textbf{f} (face) kulcsszóval egymáshoz rendeli ezeket. A lap utasításban például a lapot négy csúcsponttal adtuk meg (ez egy négyszög), és / jellel elválasztva minden csúcshoz közöltük a csúcspont, a textúra pont és a normálvektor sorsszámát. Most következzen pár gyakran előforduló face definíció:\\

\noindent Csak vertexekből áll a felület.
\begin{cpp} 
f v1 v2 v3 v4 ...
\end{cpp}
Minden vertexhez hozzá van rendelve egy textúra koordináta is.
\begin{cpp} 
f v1/vt1 v2/vt2 v3/vt3 ...
\end{cpp}
Minden vertex tartalmaz textúra koordinátát és normál vektort.
\begin{cpp} 
f v1/vt1/vn1 v2/vt2/vn2 v3/vt3/vn3 ...
\end{cpp}
A vertexekhez csak a felületi normális lett hozzárendelve, a textúra pont opcionális.
\begin{cpp} 
f v1/ /vn1 v2/ /vn2 v3/ /vn3 ...
\end{cpp}
Összefoglalva, az OBJ formális nyelv kulcsszavakból (<\textbf{v}>, <\textbf{vt}>, <\textbf{vn}>, <\textbf{f}>), speciális karakterekből (<\textbf{/}>) és számokból (<\textbf{Float}>, <\textbf{Integer}>) épül fel. Az OBJ nyelvtanát következő szabályokkal adhatjuk meg:\\

\noindent(OBJFile) = {(Vertex)}+{(VertexTexture)}+{(VertexNormal)}+{(Face)}\\
(Vertex) = (v)+(Float)+(Float)+(Float)\\
(VertexTexture) = (vt)+(Float)+(Float)\\
(VertexNormal) = (vn)+(Float)+(Float)+(Float)\\
(Face) = (f)+{(VertexOfFace)}\\
(VertexOfFace) = (Integer)+[(/)+[(Integer)]+[(/)+[(Integer)]]]\\

A szögletes zárójel [] az opcionalitás jele, azaz a benne foglalt fogalom egyszer vagy egyszer sem jelenik meg. A teljes OBJ formátum nagyon komplex, sok olyan eszköz van benne, ami egy „egyszerű” objektum modellezéséhez feleslegesnek mondható. Ezért az itt bemutatásra kerülő példaalkalmazás csak egy olyan minimális, OBJ fájlokkal szembeni támogatást fog nyújtani, mely egy szimpla objektum megjelenítéséhez elegendő.
%\Section{Obj elemző szoftverek}
%\Section{Obj Bickbucket}
%\Section{Obj Bickbucket teszt}