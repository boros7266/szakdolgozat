\Chapter{Bevezetés}

Dolgozatom témáján sokat gondolkoztam, hogy vajon mi lenne a legmegfelelőbb számomra, mind saját elvárásaimmal kapcsolatban mind személyes fejlődésem érdekében. Fontosnak tartottam a választásommal kapcsolatban, hogy a témám legyen korszerű és legyen valamilyen grafikai vonatkozása.

Így végül konzulensemmel közös megállapodásra jutva a Wavefront OBJ formátumú modelleket javító szoftver elkészítésénél maradtunk. Ennek az eszköznek a megtervezése és elkészítése rendkívül izgalmas és sok érdekes programozási problémát rejt magában.

A dolgozatom feladatként egy olyan megoldás után kutatok a fent említett témakörön belül, mely segítségével elsősorban elemezni lehet a különböző OBJ fájlokat, de emelett meg is lehessen azokat jeleníteni, ráadásul lehetőség legyen  a javításukra is a programom használatával. A programom felhasználási területét tekintve elsősorban grafikus megjelenítőknél például játékszoftver  vagy különböző szimulátorok megvalósításánál lehetne hasznos.

Az alábbiak voltak a kitűzött célok a szoftverrek kapcsolatban.
\begin{itemize}
\item A szoftver könnyen átlátható és kezelhető legyen, az adott témában nem szakképzett
személy számára is.
\item Legyen elég komplex ahhoz, hogy  az igényeket maximálisan kiszolgálja.
\item Felhasználóbarát felülettel rendelkezzen, illetve a funkciók lehetőségekhez mérten egyszerűen elérhetők legyenek.
\end{itemize}

A dolgoztom elején rövid ismertetők formájában bemutatásra kerülnek a témához kapcsolódó fogalmak, mint a Wavefront által tervezett fájlformátum, anyagjellemzők, textúrák. Ezek mellett bemutatásra kerülnek a hasonló témakörben használatos szoftverek. Ezt követik a program irásához felhasznált különböző technológiák, majd a tervezés lépéseinek prezentálása és a program implementációjának részletes bemutatása, bemutatva a saját fejlesztésű \texttt{obj\_corrector\_tool} csomag felépítését és a program szerkezetét. Végül, a program tesztelésének bemutatása különböző fájlok felhasználásával zárja a dolgozatot.

% A fejezet célja, hogy a feladatkiírásnál kicsit részletesebben bemutassa, hogy miről fog szólni a dolgozat.
% Érdemes azt részletezni benne, hogy milyen aktuális, érdekes és nehéz probléma megoldására vállalkozik a dolgozat.

% Ez egy egy-két oldalas leírás.
% Nem kellenek bele külön szakaszok (section-ök).
% Az irodalmi háttérbe, a probléma részleteibe csak a következő fejezetben kell belemenni.
% Itt az olvasó kedvét kell meghozni a dolgozat többi részéhez.

